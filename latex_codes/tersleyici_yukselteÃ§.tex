
\documentclass[border=10pt]{standalone}
\usepackage{circuitikz}
\begin{document}
\begin{circuitikz}[scale=1.2]

% Tersleyici Yükselteç Devresi
\draw (0,0) node[op amp] (opamp) {};

% Giriş direnci R1
\draw (opamp.-) to[R, l=<<R1>>, *-] ++(-2,0) coordinate (input);
\draw (input) to[short] ++(-0.5,0) node[left] {$V_{in}$};

% Geri besleme direnci R2  
\draw (opamp.-) to[short] ++(0,1.5) coordinate (fb1);
\draw (fb1) to[R, l=<<R2>>] ++(2.5,0) coordinate (fb2);
\draw (fb2) to[short] ++(0,-1.5) coordinate (fb3);
\draw (opamp.out) to[short, *-] (fb3);

% Çıkış
\draw (opamp.out) to[short] ++(1,0) node[right] {$V_{out}$};

% Non-inverting input to ground
\draw (opamp.+) to[short] ++(0,-0.5) node[ground] {};

% Kazanç formülü
\node[below=1cm] at (opamp.south) {$A_v = <<GainFormula>> = <<GainValue>>$};

\end{circuitikz}
\end{document}

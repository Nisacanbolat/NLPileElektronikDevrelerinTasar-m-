\documentclass[border=10pt]{standalone}
\usepackage{circuitikz}
\usepackage{siunitx}
\begin{document}
\begin{circuitikz}[american voltages]

    % Opamp
    \draw (0,0) node[op amp] (opamp) {};

    % Giriş -> R1 -> opamp + ve C1 toprağa
    \draw (-4,0) to[sV, l=Giriş, invert] (-4,-2) node[ground] {}
          (-4,0) to[R, l=R1:<<R>>] (-2,0) -- (-1.5,0)
          (-1.5,0) -- (opamp.+)
          (-1.5,0) to[C, l=C1:<<C>>] (-1.5,-2) node[ground] {};

    % Eksi girişten geri besleme
    \draw (opamp.out) -- ++(0,1.5) -| (opamp.-);

    % Çıkış
    \draw (opamp.out) to[short, *-o] (1.5,0) node[right] {$V_o$};

    % Başlık
    \node[align=center] at (0,3.3) {\textbf{Alçak Geçiren Filtre (Pozitif Girişli)}\\ $f_c = <<Cutoff>>$};

\end{circuitikz}
\end{document}
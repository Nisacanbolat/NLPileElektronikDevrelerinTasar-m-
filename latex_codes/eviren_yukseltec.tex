\documentclass{article}
\usepackage{circuitikz}
\usepackage[turkish]{babel}
\usepackage[utf8]{inputenc}
\usepackage[T1]{fontenc}
\usepackage{geometry}
\geometry{a4paper, margin=1in}

\title{Eviren Yükselteç Devresi}
\author{OpAmp Devre Çizim Programı}
\date{\today}

\begin{document}
\maketitle

\section{Eviren Yükselteç Devresi}

Bu devre, giriş sinyalini evirerek (180° faz farkıyla) ve \R2/\R1 oranında yükselterek çıkışa verir.

\vspace{1cm}
\begin{center}
\begin{circuitikz}
    \draw (0,0) node[op amp] (opamp) {}
          (opamp.+) -- ++(-0.5,0) node[ground] {};
    \draw (opamp.-) -- ++(-0.5,0) 
          to[R, l=\R1] ++(-2,0) node[left] {$V_{in}$};
    \draw (opamp.-) -- ++(0,1.5) 
          to[R, l=\R2] ++(3,0) -- ++(0,-1.5)
          (opamp.out) -- ++(1,0) node[right] {$V_{out}$};
    \draw (opamp.out) -- ++(0.5,0) -- ++(0,-0.01) -- ++(0,0.01)
          to[short] ++(0.5,0) -- ++(0,1.5) -- ++(-3.5,0);
\end{circuitikz}
\end{center}

\section{Devre Parametreleri}
\begin{itemize}
\item R1 = \R1
\item R2 = \R2
\item Kazanç (Av) = \Gain
\end{itemize}

\end{document}
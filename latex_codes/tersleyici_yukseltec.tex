\documentclass[border=10pt]{standalone}
\usepackage{circuitikz}
\usepackage{siunitx}

\begin{document}
\begin{circuitikz}[american voltages]
    \draw (0,0) node[op amp] (opamp) {};
    
    % Giriş
    \draw (opamp.-) to[R, l_=R1:<<R1>>, -*] (-3,0.5) -- (-4,0.5) 
          to[V, l_=Giriş, invert] (-4,-1.5) node[ground] {};
    
    % Geri besleme
    \draw (opamp.-) -- ++(0,1.5) to[R, l=R2:<<R2>>] (opamp.out |- 0,1.5) -- (opamp.out);
    
    % Çıkış
    \draw (opamp.out) to[short, *-o] (1.5,0) node[right] {Çıkış};
    
    % Toprak
    \draw (opamp.+) -- ++(0,-0.5) node[ground] {};
    
    % Başlık ve kazanç bilgisi
    \node[align=center] at (0,3) {\textbf{Tersleyici Yükselteç}\\ $A_v = <<GainFormula>> = <<GainValue>>$};
\end{circuitikz}
\end{document}
\documentclass[border=10pt]{standalone}
\usepackage{circuitikz}
\usepackage{siunitx}

\begin{document}
\begin{circuitikz}[american voltages]
    \draw (0,0) node[op amp] (opamp) {};
    
    % Girişler
    \draw (opamp.-) -- ++(0,0.5) coordinate (junction);
    \draw (junction) to[R, l_=R1:<<R1>>, -o] ++(-2,0) node[left] {Giriş 1};
    \draw (junction) to[R, l_=R2:<<R2>>] ++(-2,-1) node[below] {Giriş 2} to[short, -o] ++(-0.5,0);
    
    % Rf geri besleme
    \draw (opamp.-) -- ++(0,1.5) to[R, l=Rf:<<Rf>>] (opamp.out |- 0,1.5) -- (opamp.out);
    
    % Pozitif giriş toprak
    \draw (opamp.+) -- ++(0,-0.5) node[ground] {};
    
    % Çıkış
    \draw (opamp.out) to[short, *-o] (2,0) node[right] {Çıkış};
    
    % Bilgi
    \node[align=center] at (0,3) {\textbf{Toplayıcı Yükselteç}\\ 
    Çıkış Formülü: $V_{out} = <<GainFormula>>$};
\end{circuitikz}
\end{document}
\documentclass[border=10pt]{standalone}
\usepackage{circuitikz}
\usepackage{siunitx}
\usepackage{amsmath}

\begin{document}
\begin{circuitikz}[american voltages]

    % Opamp
    \draw (0,0) node[op amp] (opamp) {};

    % V1 -> R1 -> - giriş
    \draw (opamp.-) -- ++(-0.5,0)
          to[R, l=R1:\quad<<R1>>] ++(-2,0)
          node[left] {$V_1$};

    % V2 -> R2 -> + giriş
    \draw (opamp.+) -- ++(-0.5,0)
          to[R, l=R2:\quad<<R2>>] ++(-2,0)
          node[left] {$V_2$};

    % + girişten toprağa R4
    \draw (opamp.+) -- ++(0,-1.2)
          to[R, l=R4:\quad<<R4>>] ++(0,-1) node[ground] {};

    % Geri besleme (yukarıdan sağa)
    \draw (opamp.-) -- ++(0,1.2) coordinate (fbtop)
          to[R, l=R3:\quad<<R3>>] ++(2,0) coordinate (fbright)
          -- (opamp.out);

    % Çıkış
    \draw (opamp.out) to[short, *-o] ++(1.5,0) node[right] {$V_{out}$};

    % Başlık ve kazanç
    \node[align=center] at (0,3.2) {\textbf{Fark Yükselteç} \\[3pt] $A_v = \dfrac{R_3}{R_1} = <<GainValue>>$};

\end{circuitikz}
\end{document}

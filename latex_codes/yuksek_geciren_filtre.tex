\documentclass[border=10pt]{standalone}
\usepackage{circuitikz}
\usepackage{siunitx}

\begin{document}
\begin{circuitikz}[american voltages]

    % Opamp yerleşimi
    \draw (2,0) node[op amp] (opamp) {};

    % Giriş sinyali kaynağı
    \draw (-3,0) to[sV, l=Giriş, invert] (-3,-2) node[ground]{};

    % Girişten kondansatöre
    \draw (-3,0) to[C, l=C1:<<C>>] (-1.5,0) -- (opamp.+);

    % Direnç toprağa
    \draw (-1.5,0) to[R, l=R1:<<R>>] (-1.5,-2) node[ground]{};

    % Geri besleme
    \draw (opamp.out) -- ++(0,1.5) -| (opamp.-);

    % Çıkış
    \draw (opamp.out) -- ++(2,0) node[right] {$V_o$};

    % Başlık
    \node[align=center] at (1,3.2) {\textbf{Yüksek Geçiren Filtre (Pozitif Girişli)}\\
    $f_c = <<Cutoff>>$};

\end{circuitikz}
\end{document}

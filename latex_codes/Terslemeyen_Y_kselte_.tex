\documentclass[border=10pt]{standalone}
\usepackage{circuitikz}
\usepackage{siunitx}

\begin{document}
\begin{circuitikz}[american voltages]
    % Opamp'ı merkeze yerleştir
    \draw (0,0) node[op amp] (opamp) {};
    
    % Giriş bağlantısı - Opamp'ın pozitif girişine
    \draw (opamp.+) to[short, -o] (-2,0) node[left] {Giriş};
    
    % Negatif geri besleme devresi
    \draw (opamp.-) -- ++(0,1) coordinate (feedback) 
          to[R, l_=R1:10.00 kΩ] ++(-2,0) node[ground] {};
    \draw (feedback) to[R, l=R2:90.00 kΩ] ++(2,0) -| (opamp.out);
    
    % Çıkış bağlantısı
    \draw (opamp.out) to[short, *-o] (2,0) node[right] {Çıkış};
    
    % Başlık ve kazanç bilgisi
    \node[align=center] at (0,3) {\textbf{Terslemeyen Yükselteç}\\ 
    $A_v = <<GainFormula>> = <<GainValue>>$};
\end{circuitikz}
\end{document}
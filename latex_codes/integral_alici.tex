\documentclass[border=10pt]{standalone}
\usepackage{circuitikz}
\usepackage{siunitx}

\begin{document}
\begin{circuitikz}[american voltages]

    % Opamp yerleşimi
    \draw (3,0) node[op amp] (opamp) {};

    % X noktası
    \coordinate (X) at (0,0);

    % Giriş kaynağı → R → X
    \draw (-3,0) to[sV, l=Giriş, invert] (-3,-2) node[ground]{};
    \draw (-3,0) to[R, l=R:<<R>>] (X);

    % X → opamp.-
    \draw (X) -- (opamp.-);

    % Geri besleme C (opamp.out → yukarı → sola → X)
    \draw (opamp.out) -- ++(0,2) 
          to[C, l=C:<<C>>] (0,2) -- (X);

    % + giriş → toprak
    \draw (opamp.+) -- ++(0,-1) node[ground]{};

    % Çıkış
    \draw (opamp.out) -- ++(2,0) node[right] {$V_o$};

    % Başlık ve formül
    \node[align=center] at (1.5,3.5) {\textbf{İntegral Alıcı Devresi}\\
    $<<Formula>>$};

\end{circuitikz}
\end{document}
